\documentclass{article}
\nonstopmode

\usepackage{color}
\usepackage{parskip}
\usepackage{polski}
\usepackage[utf8]{inputenc}
\usepackage{amsthm}
\usepackage{amsmath}
\usepackage{amsfonts}

\renewcommand{\familydefault}{\sfdefault} 

\reversemarginpar

\title{Jajo w czopku}
\author{Paweł Czerwniński, Nikolaos Dymitriadis, Janusz Korwin-Mikke}

\begin{document}
\maketitle

\section{Języki}
	\textbf{def.} \textit{alfabetem} nazywamy dowolny skończony zbiór $\Sigma$.
	
	\textbf{def.} \textit{słowem} nad alfabetem $\Sigma$ nazywamy każdy skończony ciąg $w = w_1 w_2 ... w_n$, gdzie $w_1,..., w_n \in \Sigma$. Ciąg pusty nazywamy \textit{słowem pustym} i oznaczamy $\epsilon$. Zbiór wszystkich słów nad $\Sigma$ oznaczamy $\Sigma^*$, zaś zbiór słów niepustych $\Sigma^+$.
	
	\textbf{def.} \textit{językiem} nazywamy dowolny podzbiór $L \subseteq \Sigma^*$, czyli pewien zbiór słów nad $\Sigma$.
	
	\textbf{def.} \marginpar{wyrażenia regularne} \textit{standardowe wyrażenia regularne} to wyrażenia opisujące język przy użyciu:
	\begin{itemize}
		\item stałych
		\item konkatenacji
		\item sumy
		\item operatora $^*$
	\end{itemize}
	
	\textbf{def.} \textit{rozszerzone wyrażenia regularne} - wyrażenia regularne rozszerzone o:
	\begin{itemize}
		\item dopełnienie
		\item przecięcie
	\end{itemize}
	
	\textbf{def.} \marginpar{języki regularne}\textit{języki regularne} - języki możliwe do opisania przy użyciu wyrażeń regularnych.
	
	\textbf{tw.} Jeżeli $L \subseteq 1^*$(\textit{język unarny}), to $L*$ jest językiem regularnym.
	
	\textbf{tw.} przecięcie i dopełnienie języków regularnych jest również językiem regularnym.
	
	\textbf{tw.} (lemat Higmana) Niech $v \preceq w$ oznacza, iż słowo $v$ jest podciągiem $w$. Dla każdego języka $L \subseteq \Sigma^*$ zbiór $Min_\preceq(L)$ \textit{słów minimalnych} ze względu na $\preceq$ jest skończony. To znaczy, że usuwając z każdego słowa w $L$ część znaków otrzymamy skończony podjęzyk $L' \subseteq L$.
	
	\textbf{tw.} dla dowolnego języka $L$ nad skończonym alfabetem zbiór podciągów $subs(L)$ oraz zbiór nadciągów $sup(L)$ słów języka $L$ stanowią języki regularne.
	
	\textbf{def.} Dla języka $L \subseteq \Sigma^*$ powiemy że słowa $w,w' \in \Sigma^*$ są \textit{L-równoważne}, jeśli $\forall u \in \Sigma^*$ zachodzi $wu \in L \iff w'u \in L$, co oznaczamy $w \sim_L w'$.
	
	\textbf{tw.} jeżeli dla języka $L$ relacja $\sim_L$ posiada nieskończenie wiele klas abstrakcji, to $L$ jest nieregularny.
	
	\textbf{języki nieregularne:}
		\begin{itemize}
			\item $\{a^n b^n : b \in \mathbb{N}\}$
			\item $\{a^n : a \in \mathbb{P}\}$
		\end{itemize}
	
\section{Automaty}
	Oznaczenia:
		\begin{itemize}
			\item $S$ - stany
			\item $I \subseteq S$ - stany początkowe
			\item $F \subseteq S$ - stany akceptujące
			\item $\sigma \subseteq (S \times \Sigma \times S)$ - relacja przejścia
			\item $p \rightarrow^a r$ - przejście ze stanu p do r po literze a 
			\item $p \rightarrow^w r$ - przejście ze stanu p do r po słowie w
		\end{itemize}
	
	\textbf{	def.} \marginpar{automat} \textit{automatem} nad alfabetem $\Sigma$ nazywamy czwórkę $A = (S, I, F, \sigma)$, to jest zbiór stanów z wyróżnionymi stanami początkowymi i akceptującymi oraz określoną na nich relacją przejścia. Zazwyczaj $I = \{s_0\}$ jest singletonem i utożsamiamy go z $s_0$.
	
	\textbf{def.} słowo $w = w_1 w_2 ... w_n$ \textit{przeprowadza} automat $A$ ze stanu $s_1$ do stanu $s_{n+1}$, jeżeli istnieją takie $s_2,..., s_n \in S$, że $s_1 \rightarrow^{w_1} s_2 \rightarrow^{w_2} ... \rightarrow^{w_n} s_{n+1}$, co zapisujemy $s_1 \rightarrow^w s_{n+1}$.
	
	\textbf{def.} automat \textit{akceptuje} słowo $w \in \Sigma^*$, jeśli $p \rightarrow^w q$, gdzie p jest stanem początkowym a q akceptującym.
	
	\textbf{def.} automat  $A$ \textit{rozpoznaje} język $L \subseteq \Sigma^*$, jeżeli $L$ jest równy zbiorowi wszystkich słów akceptowanych przez $A$. Oznaczamy go wtedy jako $L = L(A)$.
	
	\textbf{def.} \marginpar{determinizm} automat jest \textit{deterministyczny}, jeśli dla każdych $q \in Q$ oraz $a \in \Sigma$ istnieje dokładnie jedno $p \in Q$ takie, że $q \rightarrow^a p$. Odpowiednio, automat nie będący deterministycznym nazywamy automatem \textit{niedeterministycznym}.

	\textbf{tw.} Dla każdego automatu niedeterministycznego istnieje deterministyczny automat który akceptuje te same słowa.

	\textbf{def.} automaty deterministyczne $A$ i $B$ są \textit{homomorficzne}, jeżeli istnieje funkcja $h: S_A \rightarrow S_B$, taka że:
	\begin{enumerate}
		\item jeżeli $s_0$ jest stanem początkowym $A$, to $h(s_0)$ jest stanem początkowym $B$
		\item jeżeli $s \in S_A$ jest akceptujący, to $h(s) \in B$ również
		\item jeżeli dla $s, s' \in S_A$ zachodzi $s \rightarrow^w s'$, to zachodzi również $h(s) \rightarrow^w h(s')$.
	\end{enumerate}
	
	% co to jest to A_L
	\textbf{tw.} jeśli $A$ jest deterministycznym automatem rozpoznającym $L \subseteq \Sigma^*$ i wszystkie stany $A$ są osiągalne, to istnieje homomorfizm $A \rightarrow^h_{na} A_L$. 
	
	\textbf{tw.} \marginpar{regularność} język jest regularny wtw. jest rozpoznawany przez pewien skończony automat.

	\textbf{tw.}(lemat o pompowaniu) niech $L \subseteq \Sigma^*$ - język reguralny. Wówczas istnieje taka stała $c \in \mathbb{N}$, taka że dla każdego słowa $w \in L$, które ma $|w| >= c$, istnieje podział $w = xyz$, taki że $xy^{i}z \in L$ dla każdego $i >= 0$.

\subsection{Automaty z $\epsilon$-przejściami}	
	\textbf{def.} \textit{automat niedeterministyczny z $\epsilon$-przejściami} - automat niedeterministyczny, taki że $\sigma \subseteq (S \times (\Sigma \cup \{\epsilon\}) \times S)$. To znaczy, że automat może spontanicznie przechodzić między stanami pomiędzy kolejnymi znakami czytanego słowa, co oznaczamy $s \rightarrow s'$.
	
	\textbf{tw.} Dla każdego automatu niedeterministycznego z $\epsilon$-przejściami przejściami istnieje i można obliczyć automat deterministyczny rozpoznający ten sam język.

%		\textbf{def.} \textit{automat alternujący} - automat, w którym stany są podzielone na 2 grupy: egzystencjalne i uniwersalne:
%		\begin{itemize}
%			\item \textit{semantyka uniwersalna} - każdy możliwy stan musi być akceptującym
%			\item \textit{semantyka egzystencjalna} - istnieje jakieś przejście że trafiamy do stanu akceptującego
%		\end{itemize}

%	\textbf{Tw.} (alternujące są równoważne zwykłym) Dla każdego automatu alternującego istnieje automat deterministyczny, który akceptuje te same słowa.

%	\textbf{Lemat:} Niech A - automat alternujący. Istnieje deterministyczny B taki, że dla każdego słowa $a_1 ... a_n$ automat A akceptuje $a_1...a_n \leftrightarrow B$ akceptuje an...a1(taki łatwo zamienić na szukany)

%	\textbf{Tw.} Niech $L^*$ - dowolny język. Wówczas L jest jezykiem reguralnym wtw. A(L) ma skończoną liczbę stanów, dodatkowo $A_L$ ma minimalną liczbe stanów spośród automatów deterministycznych rozpoznających L.

\section{Obliczenia}
	\textbf{def.} Jan jest \textit{kucem}.

\end{document}
